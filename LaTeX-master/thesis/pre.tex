\section{はじめに}
\label{pre}
\hspace{1zw}


\subsection{背景}

% これはテスト用のファイルです.

\LaTeX で文章を書くとき,ソースファイルをGitでバージョン管理しながら,添削のためにPDFファイルを出力するという流れがしばしば発生する.
しかし,PDFファイルはコミットする前にコンパイルしなければ正しくバージョン管理されず,ヒューマンエラーが発生しやすい.
つまり,ソースファイルと合わせてバージョン管理していたはずのPDFファイルが,その時点において最新版ではない可能性がある.

GitHubにプッシュした時点でPDFファイルを出力するという試みがなされている\cite{raven38,denkiuo604,takuseno}.
これらはGitHub Actions\cite{github-actions}を用いて実現しており,自動化することによってヒューマンエラーの可能性を排除している.
しかし,いずれも1つのPDFファイルを出力することを想定しており,複数の出力が必要な場合には触れられていない.

\subsection{目的}

以上の背景のもと,ここでは,\LaTeX で文章を書く際に複数のPDFファイルを出力する必要がある状況を想定し,PDFファイルの出力をGitHub Actionsを用いて実現することを目的とする.
また,出力先は各リポジトリのReleasesとする.
これにより,例えば卒業論文およびその要旨をPDFファイルで出力しなければならない状況においても,自動的に複数のPDFファイルが生成された上で配置されるため,問題が起こりにくいと考えられる.